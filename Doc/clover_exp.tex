\documentclass[12pt]{article}

% Packages
\usepackage[a4paper,left=2cm,right=2cm,top=3cm,bottom=3cm]{geometry}
\usepackage{mathpazo}
\usepackage[utf8]{inputenc}
\usepackage[english]{babel}
\usepackage{amsmath,amssymb}
\usepackage{mathrsfs}
\usepackage{algpseudocode}
\usepackage{fancyhdr}

\thispagestyle{fancyplain}
\renewcommand{\headrulewidth}{0pt}
\fancyhead[R]{Fernando Romero-López, 2020 \\ fernando.romero@uv.es }

% Settings
\setlength{\parindent}{0mm}
\newcommand{\tr}{\mathrm{tr}}
\newcommand{\re}{\mathrm{Re}}

\begin{document}
\section*{Exponential Clover Term}
The exponential version of the clover term  (including mass term) can be written as
\begin{equation}
 D_{sw} = \sum_x  (4+m_0) \exp ( A(x) ), \text{ with } A(x) =  -\frac{c_{sw}}{4(4+m_0)}\sum_{\mu,\nu}\sigma_{\mu\nu}F_{\mu\nu}(x)
 \label{eq:clover}
\end{equation}

with the (unconventional) definition of $\sigma_{\mu\nu}$ given by
\begin{equation}
 \sigma_{\mu\nu} = \frac{1}{2}[\gamma_\mu,\gamma_\nu].
\end{equation}

With the Euclidean definition of the gamma matrices $\sigma_{\mu\nu}$ satisfies
\begin{equation}
 \sigma_{\mu\nu}^\dagger = \sigma_{\nu\mu} = -\sigma_{\mu\nu} = \sigma_{\mu\nu}^{-1}.
\end{equation}

For the Hermitian Dirac operator $\gamma^5D$ we can make the following replacement without affecting any of the calculations presented here.
\begin{equation}
 \sigma_{\mu\nu} \to \bar{\sigma}_{\mu\nu} = \gamma_5\sigma_{\mu\nu}.
\end{equation}

The field strength tensor is defined as
\begin{equation}
 F_{\mu\nu}(x) = \frac{1}{8}\left\{Q_{\mu\nu}(x) - Q_{\mu\nu}^\dagger(x)\right\}
\end{equation}

with
\begin{align}
\begin{split}
 Q_{\mu\nu}(x)
 &= U_\mu(x)U_\nu(x+\hat{\mu})U_\mu^\dagger(x+\hat{\nu})U_\nu^\dagger(x) \\
 &+ U_\nu(x)U_\mu^\dagger(x-\hat{\mu}+\hat{\nu})U_\nu^\dagger(x-\hat{\mu})U_\mu(x-\hat{\mu}) \\
 &+ U_\mu^\dagger(x-\hat{\mu})U_\nu^\dagger(x-\hat{\mu}-\hat{\nu})U_\mu(x-\hat{\mu}-\hat{\nu})U_\nu(x-\hat{\nu}) \\
 &+ U_\nu^\dagger(x-\hat{\nu})U_\mu(x-\hat{\nu})U_\nu(x+\hat{\mu}-\hat{\nu})U_\mu^\dagger(x)
\end{split}
\end{align}

Because $Q_{\mu\nu}^\dagger=Q_{\nu\mu}$ we have $F_{\mu\nu}=-F_{\nu\mu}$. For this reason we can change the sum over $\mu,\nu$ in Eq. \eqref{eq:clover} to a sum over $\mu<\nu$ and a factor of two.
\begin{equation}
 A(x) = -\frac{c_{sw}}{2(4+m_0)}\sum_{\mu<\nu}\sigma_{\mu\nu}F_{\mu\nu}(x)
 \label{eq:clover2}
\end{equation}

The quantity $\sigma_{\mu\nu}F_{\mu\nu}$ is Hermitian and block diagonal. It can be written as
\begin{align}
A(x)=-\frac{c_{sw}}{2(4+m_0)} \sum_{\mu<\nu}\sigma_{\mu\nu}F_{\mu\nu} = 
 \begin{pmatrix}
 a & b & 0 & 0 \\
 b^\dagger & -a & 0 & 0 \\
 0  & 0 & c & d \\
 0 & 0 & d^\dagger & -c
 \end{pmatrix} \equiv \begin{pmatrix}
 A^+ & 0 \\ 
 0 & A^- 
 \end{pmatrix}, \label{eq:blocks}
\end{align}
where $A^\pm$ are $2 \times 2$ matrices in spin space and $a,b,c,d$ are $N_F \times N_F$. This formulation of $A(x)$ as a block matrix will be useful for the exponentiation.

\subsection*{Evaluation of the operator}

The evaluation of the exponential of $A(x)$ can be split as:
\begin{equation}
\exp A(x)  = \begin{pmatrix}
\exp A^+ & 0 \\
0& \exp A^-
\end{pmatrix}
\end{equation}
and so, the problem is reduced to the exponential of two $(2 N_F) \times  (2 N_F)$ matrices. The evaluation can be performed in two ways. 
\begin{enumerate}
\item Using the Taylor expansion:
\begin{equation}
\exp(A^\pm) = \sum_{k=0}^N  \frac{1}{k!} (A^\pm)^k.
\end{equation}
\item Using the Horner scheme:
\begin{equation}
\exp(A^\pm) = \sum_{k=0}^{\dim A^\pm -1} b_k(A^\pm)  (A^\pm)^k, \label{eq:exphorner}
\end{equation}
where $b_k$ are computed recursively as follows. We start with
\begin{align}
q_{N,0} = 1/N!, q_{N,1 \cdots (\dim A^\pm)-1} = 0.
\end{align}
Then, the recursion proceeds:
\begin{align}
\begin{split}
&q_{n,0} = - p_0 q_{n+1, \dim A^\pm-1}  + 1/n!, \\  &q_{n,i} = - p_i q_{n+1, \dim A^\pm-1}  + q_{n+1,i-1}, \text{ with } i=1 \cdots (\dim A^\pm) -1 , \label{eq:horner}
\end{split}
\end{align}
where $p_i$ represent the coefficients of the characteristic  polynomial of the matrix $A^\pm$:
\begin{equation}
P(A^\pm) = \sum_{n=0}^{\dim A\pm} p_n (A^\pm)^n.
\end{equation}
For instance, the characteristic polynomial of a $4 \times 4$ traceless matrix has the following coefficients:
\begin{equation}
p_0=\frac{1}{8 } \left(\tr A^2\right)^2 - \frac{1}{4} \tr A^4 ,\ p_1 = -\frac{1}{3}\tr  A^3, \ p_2= -\frac{1}{2}\tr  A^2, \ p_3=0, \ p_3=1.
\end{equation}
Finally, the coefficients of Eq. \ref{eq:exphorner} are $b_k  =q_{0,k} $. 
\end{enumerate}
The Horner scheme method is currently implemented only for $SU(2)$ and $SU(3)$ with fundamental fermions.

\subsection*{Pseudofermion Forces}
For the forces we use the following short-hand notation for the derivative with respect to the link variables.
\begin{equation}
 \dot{S} = \partial_{x,\mu}^a S.
\end{equation}
To calculate the pseudofermion forces let us write down the action as
\begin{equation}
 S = \phi^\dagger(H^{-2})\phi,
\end{equation}

where $H=\gamma^5D$ is the Hermitian Dirac operator. When differentiating the action we obtain
\begin{equation}
 \dot{S} = -2\re~\xi^\dagger\dot{H}\eta,
 \label{eq:dotS}
\end{equation}

with the definitions
\begin{align}
 \eta &= H^{-2}\phi, \\
 \xi &= H\eta.
\end{align}

\subsubsection*{Forces}
Here we will only consider the forces from the clover term. For the exponential version of the clover term, the implementation is very similar to the traditional clover term.

The clover part of the Dirac operator is given by
\begin{equation}
 H_{sw} = (4+m_0) \gamma_5 \exp \left(- \frac{c_{sw}}{2(4+m_0)}  \sum_{\mu<\nu}{\sigma}_{\mu\nu}F_{\mu\nu}(x)  \right) = (4+m) \gamma_5 \exp A(x).
 \label{eq:Hsw}
\end{equation}

An optimized way of calculating the derivative is provided by the double Horner scheme. The basic idea is that the derivative of a matrix can be expressed as:
\begin{equation}
d e^A = \sum_{k=0}^{\dim A -1} \sum_{l=0}^{\dim A-1} C_{kl}(A) A^l dA A^k, \label{eq:dexphorner}
\end{equation} 
where  the $C_{kl}$ coefficients depend on the matrix $A$, similarly to the ones Eq. \ref{eq:exphorner}. They are calculated performing first the iteration in Eq. \ref{eq:horner}, and then repeating the iteration process on the result of the first iteration. For compactness, we shall omit the limits of the sum henceforth. When inserting Eq. \eqref{eq:Hsw} in Eq. \eqref{eq:dotS}, and using Eq. \eqref{eq:dexphorner}, we obtain
\begin{equation}
 \dot{S} = c_{sw} \sum_{k}  \sum_{\mu<\nu}\re(  \xi_k^\dagger\bar{\sigma}_{\mu\nu}\dot{F}_{\mu\nu}\eta_k),
\end{equation}
with 
\begin{equation}
\xi_k = \sum_l \begin{pmatrix}
C^+_{kl} (A^+)^l \xi^+ \\
C^-_{kl} (A^-)^l \xi^-
\end{pmatrix}, \text{ and } 
\eta_k =  \begin{pmatrix}
 (A^+)^k \eta^+ \\
(A^-)^k \eta^-
\end{pmatrix}.
\end{equation}


From the definition of $F_{\mu\nu}$ it follows that
\begin{align}
 \dot{S} &=  \frac{1}{8}c_{sw}\sum_k \sum_{\mu<\nu}\re(\xi_k^\dagger\bar{\sigma}_{\mu\nu}\dot{Q}_{\mu\nu}\eta_k + \xi_k^\dagger\bar{\sigma}_{\mu\nu}^\dagger\dot{Q}_{\mu\nu}^\dagger\eta_k), \\
 &=
 \frac{1}{8}c_{sw}\sum_k \sum_{\mu<\nu}\re(\xi_k^\dagger\bar{\sigma}_{\mu\nu}\dot{Q}_{\mu\nu}\eta_k + \eta_k^\dagger\bar{\sigma}_{\mu\nu}\dot{Q}_{\mu\nu}\xi_k).
\end{align}

This can in be written as
\begin{equation}
 \dot{S} = \frac{1}{8}c_{sw}\sum_{\mu<\nu} \re~\tr\left[\dot{Q}_{\mu\nu} \sum_k\left\{\bar{\sigma}_{\mu\nu}\eta_k(x)\otimes\xi_k^\dagger(x) + \bar{\sigma}_{\mu\nu}\xi_k(x)\otimes\eta_k^\dagger(x)\right\}\right]
\end{equation}

In a short hand notation we need to calculate
\begin{equation}
 \dot{S} = \frac{1}{8}c_{sw}\re~\tr[\dot{Q}_{\mu\nu}(x)X_{\mu\nu}(x)]
 \label{eq:force}
\end{equation}

with
\begin{equation}
 X_{\mu\nu}(x) = \sum_k \bar{\sigma}_{\mu\nu}\eta_k(x)\otimes\xi_k^\dagger(x) + \bar{\sigma}_{\mu\nu}\xi_k(x)\otimes\eta_k^\dagger(x).
\end{equation}
This matrix has the properties $(X_{\mu\nu})^\dagger=X_{\nu\mu}=-X_{\mu\nu}$. Note that the difference between the standard and exponential csw enters only in the different definition of $X_{\mu\nu}$.

 The expression for $\dot{Q}_{\mu\nu}(x)$ contains eight different terms (two from each of the four leafs). The eight contributions to the force can be written as
\begin{align}
 F_1(x) &=
 \re~\tr[\dot{U}_\mu(x)U_\nu(x+\hat{\mu})U_\mu^\dagger(x+\hat{\nu})U_\nu^\dagger(x)X_{\mu\nu}(x)] \\
 F_2(x) &=
 \re~\tr[\dot{U}_\mu(x)U_\nu^\dagger(x+\hat{\mu}-\hat{\nu})U_\mu^\dagger(x-\hat{\nu})X_{\mu\nu}^\dagger(x-\hat{\nu})U_\nu(x-\hat{\nu})] \\
 F_3(x) &=
 \re~\tr[\dot{U}_\mu(x)U_\nu^\dagger(x+\hat{\mu}-\hat{\nu})X_{\mu\nu}^\dagger(x+\hat{\mu}-\hat{\nu})U_\mu^\dagger(x-\hat{\nu})U_\nu(x-\hat{\nu})] \\
 F_4(x) &=
 \re~\tr[\dot{U}_\mu(x)X_{\mu\nu}(x+\hat{\mu})U_\nu(x+\hat{\mu})U_\mu^\dagger(x+\hat{\nu})U_\nu^\dagger(x)] \\
 F_5(x) &=
 \re~\tr[\dot{U}_\mu(x)X_{\mu\nu}^\dagger(x+\hat{\mu})U_\nu^\dagger(x+\hat{\mu}-\hat{\nu})U_\mu^\dagger(x-\hat{\nu})U_\nu(x-\hat{\nu})] \\
 F_6(x) &=
 \re~\tr[\dot{U}_\mu(x)U_\nu(x+\hat{\mu})X_{\mu\nu}(x+\hat{\mu}+\hat{\nu})U_\mu^\dagger(x+\hat{\nu})U_\nu^\dagger(x)] \\
 F_7(x) &=
 \re~\tr[\dot{U}_\mu(x)U_\nu(x+\hat{\mu})U_\mu^\dagger(x+\hat{\nu})X_{\mu\nu}(x+\hat{\nu})U_\nu^\dagger(x)] \\
 F_8(x) &=
 \re~\tr[\dot{U}_\mu(x)U_\nu^\dagger(x+\hat{\mu}-\hat{\nu})U_\mu^\dagger(x-\hat{\nu})U_\nu(x-\hat{\nu})X_{\mu\nu}^\dagger(x)]
\end{align}

where each term should be multiplied by $c_{sw}/8$. The calculation can be done efficiently by noticing that several products and terms appear in multiple places. Introduce the intermediate variables
\begin{align}
 Z_0 &= X_{\mu\nu}(x) \\
 Z_1 &= X_{\mu\nu}(x+\hat{\mu}) \\
 Z_2 &= X_{\mu\nu}(x-\hat{\nu}) \\
 Z_3 &= X_{\mu\nu}(x+\hat{\mu}-\hat{\nu}) \\
 Z_4 &= X_{\mu\nu}(x+\hat{\mu}+\hat{\nu}) \\
 Z_5 &= X_{\mu\nu}(x+\hat{\nu}) \\
 W_0 &= U_\mu^\dagger(x-\hat{\nu}) \\
 W_1 &= U_\nu(x-\hat{\nu}) \\
 W_2 &= U_\nu(x+\hat{\mu}) \\
 W_3 &= U_\mu^\dagger(x+\hat{\nu}) \\
 W_4 &= U_\nu^\dagger(x) \\
 W_5 &= U_\nu^\dagger(x+\hat{\mu}-\hat{\nu}) \\
 W_6 &= W_0W_1 \\
 W_7 &= W_2W_3 \\
 W_8 &= W_7W_4-W_5W_6
\end{align}

The total force can now be written as
\begin{equation}
 F(x) = \frac{c_{sw}}{8}\dot{U}_\mu(x)\left\{W_8Z_0 + Z_1W_8 - W_5(W_0Z_2W_1+Z_3W_6) + (W_2Z_4W_3+W_7Z_5)W_4\right\}
\end{equation}

This brings us down to a total of 15 matrix multiplications and 6 additions.

\subsubsection*{Even-odd preconditioning}

Even-odd preconditioning is particularly simple for the exponential case, since the force coming from the little determinant vanished. This can be seen because of the fact that:
\begin{equation}
\det D_{oo}  = \exp(\log \det D_{oo} ) =\exp( \tr \log D_{oo}) =  1,
\end{equation}
and so it is a constant term in the action that does not contribute to the force. 

\subsubsection*{Implementation of  $X_{\mu\nu}$  using Taylor series}

In the current version of  the code, the horner scheme is only implemeted for $SU(2)$ and $SU(3)$ with fundamental fermions. For other theories, a less efficient  --- but more flexible--- alternative is used. For this, we use the Taylor series:
\begin{equation}
dA = \sum_{k=0}^N \sum_{l=0}^{N-k} \frac{1}{(k+l+1)!} A^{k} dA A^{l},
\end{equation}
with $N$ sufficiently large. The implementation changes only in the definition of $X_{\mu\nu}$:
\begin{equation}
 X_{\mu\nu}(x) = \sum_{k=0}^N \bar{\sigma}_{\mu\nu}\eta_k(x)\otimes\xi_k^\dagger(x) + \bar{\sigma}_{\mu\nu}\xi_k(x)\otimes\eta_k^\dagger(x),
\end{equation}
where now:
\begin{equation}
\xi_k = \sum_l \frac{1}{(k+l+1)!} \begin{pmatrix}
(A^+)^l \xi^+ \\
 (A^-)^l \xi^-
\end{pmatrix}, \text{ and } 
\eta_k =  \begin{pmatrix}
 (A^+)^k \eta^+ \\
(A^-)^k \eta^-
\end{pmatrix}.
\end{equation}
\end{document}
