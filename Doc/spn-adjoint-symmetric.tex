\documentclass[a4paper,10pt]{article}
\usepackage[utf8]{inputenc}
\usepackage{amssymb} 
\newcommand*{\QEDB}{\null\nobreak\hfill\ensuremath{\square}}


%opening
\title{A not too concise proof of the
equivalence of the symmetric and the adjoint representations of $SP(2N)$
}
\author{}

\begin{document}

\maketitle

\begin{abstract}
    The result follows from $ - \Omega U \Omega = U^\dagger$ (where $U$ is a
    member of $SP(2N)$) and from the fact that  $ T\Omega $ (where $T$ is any
    generator of $SP(2N)$ is a symmetric matrix.
\end{abstract}

\section{Intro}

For a matrix $U$ belonging to $SU(2N)$ we have 
\begin{equation}
U^\dagger U=1.
\label{sun}
\end{equation}
The $SP(2N)$ subgroup is defined as the subset of $SU(2N)$ that satisfies
\begin{equation}
 U^T \Omega U = \Omega
 \label{symplectic}
\end{equation}
where 
\begin{equation}
 \Omega = \left\{
 \begin{array}{cc}
  0_{N} & 1_{N} \\
  -1_{N} & 0_{N} 
 \end{array}
 \right.
 \label{omega}
\end{equation}
Since $(U_1 U_2)^T = U_2^T U_1^T$, this is obviously still a group. \\
First of all, we write $U$ as 
\begin{equation}
 U = \left\{
 \begin{array}{cc}
  A & B \\
  C & D 
 \end{array}
 \right.
\end{equation}
With $U$ in this form Eq.\ref{sun} can be seen as an orthonormality relation 
between the set of rows represented by $(A\ B)$ and the set of rows represented 
by $(C\ D)$, and similarly between columns. 
Expanding Eq.\ref{symplectic}, and looking e.g. at the upper right 
corner, we see that $(D^\ast\ -B^\ast)$ needs 
to be ``proportional'' to $(A\ C)$, and other similar relations that lead us to
conclude that $U$ can be written in the form
\begin{equation}
 U = \left\{
 \begin{array}{cc}
  A & B \\
  -B^\ast & A^\ast
 \end{array}
 \right.
 \label{usymp}
\end{equation}
Writing the infinitesimal version of $U = 1_{2N} + i\lambda_i \tau_i$ and plugging it into 
Eq.\ref{symplectic}, it turns out that 
\begin{equation}
\tau^T \Omega + \Omega \tau = 0
\end{equation}
or
\begin{equation}
    \Omega \tau + (\Omega \tau)^T = 0
    \label{symp_generators_constraints}
\end{equation}
which means that $\tau \Omega$ is a symmetric matrix.
Moreover, in order for $U$ to be of the form in Eq. \ref{usymp},
$\tau$ must have the form
\begin{equation}
    \tau = \left\{
        \begin{array}{cc}
            \alpha & \beta \\
            \beta^\ast & -\alpha^\ast
        \end{array}
    \right. 
\end{equation}
The $SU(N)$ constraints for $\tau$ of being a Hermitean matrix 
mean that $\alpha = \alpha^\dagger$ (this ensures also that $\tau$ be traceless)
and $\beta = \beta^T$. We have $(\frac{N}{2})^2$ degrees of freedom for $\alpha$, and
$\frac{N}{2}\left(\frac{N}{2}+1\right)$ degrees of freedom for $\beta$. 
This means that the dimension of the algebra is 
\begin{equation}
 D_{adj} = \frac{N (N+1)}{2}
\end{equation}
which is curiously equal to the dimension of the space of \emph{real} 
(or \emph{imaginary}) symmetric tensors of dimension $N$, i.e. the dimension 
of the real symmetric representation.

\section{The symmetric and adjoint representations}

The transformation of a symmetric tensor can be written as
\begin{equation}
    S' = U^T S U
\end{equation}
which, expliciting the components 
\begin{equation}
    S_{a'b'}' = S_{ab} U^{a}_{a'}  U^{b}_{b'}  .
\end{equation}
Thanks just to the symmetricity of S, we have 
\begin{equation}
    S_{a'b'}' = S_{ab} \left( \frac{U^{a}_{a'}  U^{b}_{b'} + U^{a}_{b'}  U^{b}_{a'} } {2}\right) .
\end{equation}
which defines the symmetric representation.
For the adjoint representation, we have 
\begin{equation}
    A' = U^\dagger A U .
    \label{adjoint}
\end{equation}
where $A$ and $A'$ are linear combinations of the generators of $SP(2N)$.
Multiplying both sides of Eq.\ref{symplectic} by $\Omega^T$ ( which is also
$-\Omega$ and $\Omega^-1$) and remembering the unitarity of $U$ we have 
\begin{equation}
    U^\dagger = \Omega^T U^T \Omega
\end{equation}
Plugging this into Eq.\ref{adjoint} we have 
\begin{equation}
    \Omega A' = U^T \Omega A U
\end{equation}
Now, from Eq.\ref{symp_generators_constraints} we know that $\Omega A$ is a 
symmetric matrix.
Moreover, we know that the adjoint representation is real, because for a real
(imaginary)
$A_\pm$ (e.g., $A_\pm = B \pm B^\dagger$ with $B$ in the $SP(2N)$ algebra) we have 
that 
\begin{equation}
    A_\pm' = U^\dagger A_\pm U .
\end{equation}
is real (imaginary) as well.  \\
We also know that the space spanned by $\Omega A$ has the 
same dimension of the space of \emph{real} (or \emph{imaginary}) symmetric 
tensors in dimension $N$. 

We can thus conclude that for $SP(2N)$ the complex symmetric representation is
reducible into two copies of the adjoint representation, which is real.

\QEDB
\end{document}
