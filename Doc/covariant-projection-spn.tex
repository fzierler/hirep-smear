\documentclass[a4paper,10pt]{article}
\usepackage[utf8]{inputenc}
\usepackage{amssymb}
\newcommand*{\QEDB}{\null\nobreak\hfill\ensuremath{\square}}


% opening
\title{On the correctness of the covariant projection on group for
  SP(2N)
}
\author{}

\begin{document}

\maketitle

\begin{abstract}
  The algorithm for the covariant projection on SU(N) described on page 3 of
  arXiv:hep-lat/9304011v1 needs no changes in order to work for SP(N).
\end{abstract}

\section{Description of the algorithm for SU(N)}
Take the set of paths on a lattice starting on a given point $P_{S}$ and ending on
another point $P_{E}$.
Take the parallel transport operators corresponding to those paths.
A linear combination $M$ of those operators do not belong to the gauge group,
but it transforms covariantly under a gauge transformation.

There is an algorithm to obtain an $U$ belonging to the gauge group
that is \emph{covariant} with $M$, that is: if
\begin{equation}
  U = f(M)
\end{equation}
and
\begin{equation}
  M'= G_{1}^{\dagger} M G_{2} ,
\end{equation}
then
\begin{equation}
  U' = f(M') = G_{1}^{\dagger} f(M) G_{2} \ .
\end{equation}
This algorithm is described in arXiv:hep-lat/9304011v1, and amounts to the
following steps.
\begin{enumerate}
  \item Given any nonsingular matrix $M$, $\mathcal{H} = M^{\dagger}M$ is
    Hermitean.
    \label{step1}
  \item Notice that if $M = H U$, with $U$ unitary, then
    $\mathcal{H}=H^{\dagger}H$.
    \label{step2}
  \item We can diagonalize $\mathcal{H}$ with a unitary transformation $D$, so
    that $\mathcal{H}= D^{\dagger} \Delta D$ with $\Delta$ real.
    \label{step3}
  \item We can choose $H = D^{\dagger} \Delta^{\frac{1}{2}} D$, as $H^{\dagger} = H$.
    \label{step4}
  \item This leads us to a unitary $U = H^{-1} M$, where the inverse of $H$ can
    be easily computed as $D^{\dagger} \Delta^{-\frac{1}{2}} D$.
  \label{step5}
\end{enumerate}
Notice that if $M$ is already unitary, $U=M$. Also, for
$M' = G_{1}^{\dagger}M G_{2}$, we obtain a $H' = G_{2}^{\dagger} H G_{2}$ which
leads to $U'=G_{1}^{\dagger}U G_{2}$, as needed.

\section{The SP(N) case}

A matrix $S$ belongs to $SP(N)$ if it is a $SU(N)$ matrix and if
$S^{T} \Omega S = \Omega$. If $\Omega$ is a matrix of the form
\begin{equation}
  \Omega = \left\{
    \begin{array}{cc}
      0_{N} & 1_{N} \\
      -1_{N} & 0_{N}
    \end{array}
  \right.
  \label{omega}
\end{equation}
In addition to belonging to $SU(N)$, $S$ must also have the form
\begin{equation}
  S = \left\{
    \begin{array}{cc}
      A & B \\
      -B^\ast & A^\ast
    \end{array}
  \right. .
  \label{ssymp}
\end{equation}
(see, e.g., \verb|spn-adjoint-symmetric.tex| for a brief justification of this).
It is obvious that matrices of the form in Eq.(\ref{ssymp}) form a vector space.
It can be also verified easily that nonsingular matrices of the form in
Eq.(\ref{ssymp}) do form a group - meaning that if two matrices $S$ and $T$ are in
that form, their product is as well. We call the algebra of matrices in the form
Eq.(\ref{ssymp}) $\mathcal{A}$.

This means that linear combination of $SP(N)$ parallel transport operators still
belong to $\mathcal{A}$.

\begin{enumerate}
  \item If we apply the algorithm described for $SU(N)$ to a matrix
    $M \in \mathcal{A}$, we obtain also a Hermitean $\mathcal{H}\in \mathcal{A}$.
  \item $\mathcal{H}$ can be diagonalized with a unitary transformation $D$, but
    we can show that $D \in \mathcal{A}$. Indeed, if $w = (V ,W)$ is an
    eigenvector of $\mathcal{H}$ we can easily prove that the vector
    $\tilde{w} = (-W^{\ast},V^{\ast})$ is another eigenvector with the same
    eigenvalue.

    Note that if $\tilde{w} = k w$, then $w$ is zero.

    This means that also $D \in \mathcal{A}$.
  \item This means that also $H \in \mathcal{A}$, which in turn means that
    $U \in \mathcal{A}$. In addition to the fact that $U \in SU(N)$, this means
    that $U \in SP(N)$.
\end{enumerate}

Moreover, the fact that the algorithm is covariant for $SU(N)$ and
$SP(N) \subset SU(N)$, means that the algoriwhm is covariant for $SP(N)$ as well.

\QEDB
\end{document}
